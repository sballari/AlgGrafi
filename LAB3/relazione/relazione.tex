\documentclass{article}
\usepackage[utf8]{inputenc}
\usepackage{graphicx}
\usepackage{amsmath}
\usepackage{titling}
\usepackage{pdflscape}
\usepackage[export]{adjustbox}
\usepackage{float}
\usepackage{booktabs} % To thicken table lines
\usepackage{adjustbox}

\setlength{\droptitle}{-10em}
\pagestyle{empty}

\title{Lab. 3 - Il Commesso Viaggiatore}
\author{Ballarin Simone, Gobbo Alessio, Rossi Daniel}
\date{May 2019}

\begin{document}
\maketitle

\section*{Domanda 1}
\begin{center}
	\scalebox{0.7}{%
		\begin{tabular}{l|c|c|c|c|c|c|c|c|c}
			\toprule
			{} & \multicolumn{3}{c}{Held-Karp} & \multicolumn{3}{c}{Euristica costruttiva} & \multicolumn{3}{c}{2-approssimato} \\
			\midrule
												
			Istanza       & soluzione & tempo (s) & $\epsilon$ (\%)    & soluzione & tempo (s) & $\epsilon$ (\%)  & soluzione & tempo (s) & $\epsilon$ (\%)  \\
			\midrule
			burma14.tsp   & 3323      & 7.40E-01  & 0.0\%     & 3346      & 1.80E-04  & 0.69\%  & 4062      & 2.10E-04  & 22.24\% \\
			ulysses22.tsp & 8649      & 1.00E+01  & 23.33\%   & 7448      & 3.26E-04  & 6.2\%   & 8132      & 3.73E-04  & 15.96\% \\
			eil51.tsp     & 1065      & 1.00E+01  & 150.0\%   & 453       & 1.11E-03  & 6.34\%  & 599       & 1.11E-03  & 40.61\% \\
			kroD100.tsp   & 152544    & 1.00E+01  & 616.37\%  & 23519     & 3.85E-03  & 10.45\% & 27296     & 3.60E-03  & 28.19\% \\
			gr229.tsp     & 184720    & 1.00E+01  & 37.23\%   & 146473    & 2.05E-02  & 8.82\%  & 170205    & 1.44E-02  & 26.45\% \\
			d493.tsp      & 112558    & 1.00E+01  & 221.58\%  & 38422     & 9.06E-02  & 9.77\%  & 45330     & 6.42E-02  & 29.51\% \\
			dsj1000.tsp   & 553100651 & 1.00E+01  & 2864.15\% & 20746015  & 3.97E-01  & 11.18\% & 25728577  & 2.31E-01  & 37.88\% \\
												
			\bottomrule
		\end{tabular}}
\end{center}
\section*{Domanda 2}
Descriviamo il comportamento di ciascun algoritmo al variare delle istanze di seguito.
\subsection*{Held-Karp}
L'algoritmo esatto data la sua elevata complessità, trova la soluzione esatta entro il tempo limite di x secondi solamente nell'istanza più piccola (burma14). Sempre con il limite di x secondi vediamo come il risultato di ulysses22 possa essere considerato accettabile, mentre per le altre istanze del problema, in generale, l'errore è più significativo. Nonostante quanto appena detto, in alcuni casi, l'errore di istanze più grandi risulta inferiore a quello calcolato per istanze più piccole (gr229,d493). Questa peculiarità trova spiegazione, secondo noi, nella particolare configurazione dei pesi di queste istanze: l'ordine con cui iteriamo i circuiti possibili è deterministico. In questi due particolari problemi, casualmente una soluzione buona si trova nei primi circuiti analizzati, portando ad avere soluzioni difficilmente migliorabili aumentando di poco il tempo.

\subsection*{Euristica costruttiva}
L'euristica costruttiva scelta è stata la \textit{Random Insertion}, questa euristica prevede la costruzione di un circuito attraverso l'aggiunta iterativa di un nodo. 
Questo viene scelto casualmente ed inserito nella posizione che minimizzi la deviazione necessaria.\\
I risultati ottenuti sono sorprendentemente buoni. Tranne per il caso burma14 è l'algoritmo che trova sempre la soluzione migliore nel tempo più breve. Addirittura nell'istanza più grande l'errore risulta pari a 11\% calcolato in un tempo di 3.97E-01 secondi.\\
Al fine di ottenere una maggiore comprensione dei risultati elaborati in relazione alla natura casuale dell'algoritmo, abbiamo deciso di eseguire per ciascuna istanza del problema \textbf{10000} volte l'algoritmo. Abbiamo quindi estrapolato l'errore minimo e massimo per ciascuna istanza, riportando il tutto nella tabella a seguire:\\ 

\begin{table}[H]
	\begin{tabular}{lcc}
		\toprule
		Istanza       & $\epsilon_{min}(\%)$ & $\epsilon_{max}(\%)$ \\
		\midrule
		burma14.tsp   & 0.69                 & 12.97                \\
		ulysses22.tsp & 1.11                 & 10.11                \\
		eil51.tsp     & 1.17                 & 8.22                 \\
		kroD100.tsp   & 6.03                 & 9.92                 \\
		gr229.tsp     & 7.29                 & 13.38                \\
		d493.tsp      & 8.33                 & 11.22                \\
		dsj1000.tsp   & 10.73                & 13.84                \\
		\bottomrule
	\end{tabular}
\end{table}

\noindent Dalla tabella notiamo come gli errori massimi siano simili, mentre gli errori minimi sono lineari all'aumentare del numero di nodi dell'istanza.

\subsection*{2-approssimato}
L'algoritmo basato sull'albero di cammino minimo sembra non presentare nessuna particolare correlazione tra l'errore e la grandezza del problema.

\subsection*{Considerazioni generali}
In generale si nota che l'istanza kroD100.tsp in relazione alla sua grandezza limitata è sempre il problema che negli algoritmi approssimati da i risultati peggiori.\\
L'algoritmo in genere migliore è sempre l'euristica costruttiva in termini di errore percentuale. Per quanto riguarda l'efficienza notiamo come l'euristica costruttiva per istanze piccole sia migliore mentre per istanze più grandi risulti più efficiente l'algoritmo 2-approssimato.

\end{document}
