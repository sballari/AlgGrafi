\documentclass{article}
\usepackage[utf8]{inputenc}

\title{Lab1}
\author{Ballarin Simone }
\date{March 2019}

\begin{document}


\maketitle

\section{Introduction}
\section{Domanda 1}
Il grafo reale era formato da un numero di nodi pari a 6474, inoltre vi erano presenti circa 13000 archi. Il grafo fornito era non orientato. 
Ci viene richiesto di generare grafi casuali con medisime caratteristiche di cardinalitlà del grafo reale utilizzando gli algoritmi ER e UPA.
Al fine di attuare la richiesta abbiamo dovuto calcolare la probabilità p per l'algoritmo ER, mentre il parametro m dell'algoritmo UPA è stato fornito nella domanda stessa.
La probabilità p è stata calcolata come segue:
Sia X una variabile aleatoria come segue X=Bin((n*(n-1)/2,p), questa varibile descrive il numero di successi dell'algoritmo ER (archi scelto), il quale effettua n*(n-1)/2 scelte che rappresenano la creazione di un arco tra due nodi non identici con probabilita' p ignota.
Volendo avere una media di archi generati da ER pari a quella del grafico reale (che significa E(X)=13xxx) e inoltre sapendo E(X)=n*(n-1)/2*p, possiamo dedurre p=13xxx/n*(n-1)*2.


\section{Domanda 2}
Osservando il grafico dell'attacco casuale, si piu' osservare che dopo aver rimosso il 20 percento dei
\end{document}
